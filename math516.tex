\documentclass[a4paper, 11pt]{article}
\usepackage{fullpage}
\usepackage{titling}
\usepackage{graphicx}
\usepackage{xcolor}
\usepackage{amsmath}
\usepackage{amssymb}
\usepackage{amsthm}
\usepackage{hyperref}
\usepackage{IEEEtrantools} 
\usepackage{bbm}
\usepackage{lineno}
\usepackage{multirow}
\usepackage{longtable}
\usepackage{makecell}
\usepackage{lipsum}
\usepackage{etoolbox} %% <- for \pretocmd and \apptocmd
\usepackage{algorithm}
\usepackage[noend]{algpseudocode}
\usepackage{natbib}
\newcommand{\cP}{\mathcal{P}}
\newcommand{\R}{\mathbb{R}}
\newcommand{\Q}{\mathbb{Q}}
\newcommand{\N}{\mathbb{N}}
\newcommand{\cZ}{\mathcal{Z}}
\newcommand{\cX}{\mathcal{X}}
\newcommand{\cM}{\mathcal {M}}
\newcommand{\tadashi}[1]{\textcolor{blue}{\textbf{[Tadashi: }#1\textbf{]}}}
\newcommand{\shuai}[1]{\textcolor{blue}{\textbf{[Shuai: }#1\textbf{]}}}
\newcommand{\suminf}{\sum_{n=1}^\infty}
\newcommand{\dist}{\text{dist}}

\title{Math 516: Linear Analysis}
\author{Shuai Liu}
\makeatletter %% <- make @ usable in macro names
\newcommand*\linenomathpatch{\@ifstar{\linenomathpatch@AMS}{\linenomathpatch@}}
\newcommand*\linenomathpatch@[1]{
  \expandafter\pretocmd\csname #1\endcsname {\linenomathWithnumbers}{}{}
  \expandafter\pretocmd\csname #1*\endcsname{\linenomathWithnumbers}{}{}
  \expandafter\apptocmd\csname end#1\endcsname {\endlinenomath}{}{}
  \expandafter\apptocmd\csname end#1*\endcsname{\endlinenomath}{}{}
}
\newcommand*\linenomathpatch@AMS[1]{
  \expandafter\pretocmd\csname #1\endcsname {\linenomathWithnumbersAMS}{}{}
  \expandafter\pretocmd\csname #1*\endcsname{\linenomathWithnumbersAMS}{}{}
  \expandafter\apptocmd\csname end#1\endcsname {\endlinenomath}{}{}
  \expandafter\apptocmd\csname end#1*\endcsname{\endlinenomath}{}{}
}
\let\linenomathWithnumbersAMS\linenomathWithnumbers
\patchcmd\linenomathWithnumbersAMS{\advance\postdisplaypenalty\linenopenalty}{}{}{}
\makeatother %% revert @
\linenomathpatch{IEEEeqnarray}
\linenomathpatch{equation}
\linenomathpatch*{gather}
\linenomathpatch*{multline}
\linenomathpatch*{align}
\linenomathpatch*{alignat}
\linenomathpatch*{flalign}
\newtheorem{theorem}{Theorem}[section]
\newtheorem{lemma}[theorem]{Lemma}
\newtheorem{corollary}[theorem]{Corollary}
\newtheorem{remark}[theorem]{Remark}
\newtheorem{fact}[theorem]{Fact}
\newtheorem{property}[theorem]{Property}
\newtheorem{claim}{CLAIM}
\newtheorem{proposition}[theorem]{Proposition}
\theoremstyle{definition}
\newtheorem{definition}[theorem]{Definition}
\newtheorem{condition}[theorem]{Condition}
\newtheorem{example}[theorem]{Example}
\newtheorem{assumption}{Assumption}
\newtheorem{exercise}[theorem]{Exercise}
\begin{document}
\maketitle
\section{Quotient Spaces}
\begin{definition}
    Let $X$ be a vector space and $Y$ be a subspace. Define $\hat x$ be $\{z: z=x+y, y\in Y\}$. Then $X/Y = \{\hat x: x\in X\}$ is a vector space. 
    $Q:X \to X/Y$ is a canonical quotient map. Suppose in addition that $X$ is a normed space and $Y$ is a closed subspace, then we can make $X/Y$ into a normed space:
    \begin{equation}
        \|\hat x\| = \inf_{z \in \hat x}{\|z\|} = \inf_{x}{\|x-y\|}=\text{dist}(x,Y)
    \end{equation}
\end{definition}
$X/Y$ is indeed a vector space. If $\|\hat x\|=0$ then $\text{dist}(x,Y)=0$ and $Y$ is closed implies $x\in Y$. Therefore $\hat x=Y$ which is the 0 element in quotient space.
$\forall a,b \in X$, $\hat a+\hat b = Qa+Qb=Q(a+b)=\{z:z=a+b+y, y\in Y\}$. For all $\epsilon >0 $, there exists $y_0, y_1$ such that $\|a-y_0\|\le \|\hat a\|+\epsilon/2$ and
$\|b-y_0\|\le \|\hat b\|+\epsilon/2$. Then
\begin{equation}
    \|\hat a +\hat b\| = \inf_{y\in Y}\|a+b-y\|\le \|a+b-(y_0+y_1)\|\le \|a-y_0\| + \|b-y_1\|\le \|\hat a\|+\epsilon/2 + \|\hat b\|+\epsilon/2\le \|\hat a\|+\|\hat b\|+\epsilon
\end{equation}
Since $\epsilon$ can be arbitrarily small, we have $ \|\hat a +\hat b\|\le \|\hat a\|+\|\hat b\|$.
$\|\alpha \hat a\| = |\alpha|\|\hat a\|$ is trivial.
\begin{exercise}\label{ex:Q}
    $\|Q\|=1, Q(B_x^\circ)=B_Y^\circ$
\end{exercise} 
\begin{theorem}
    If $X$ is a Banach space and $Y$ is a closed subspace of $X$. Then $X/Y$ is a Banach space. 
\end{theorem}  
\begin{proof}
    It is sufficient to prove that all absolutely convergent series are all convergent. Let $\sum_{n=1}^\infty \hat x_n$ be an absolutely convergent subsequence,
    i.e. $\sum_{n=1}^\infty \|\hat x_n\|<\infty$. For all $\hat x_n$, pick a representative $z_n$ such that $\|z_n\|\le 2\|\hat x_n\|$ and note that $Qz_n = \hat x_n$.
    Then $\sum_{n=1}^\infty \|z_n\|\le 2\sum_{n=1}^\infty \|x_n\|<\infty$. Since $X$ is Banach, $z:=\suminf z_n$ is a convergent sequence. By exercise \ref{ex:Q}, we know that $Q$ is a bounded operator therefore continuous.
    Then $\suminf \hat x_n = \suminf Qz_n=Qz$ converges
\end{proof}
\begin{theorem}[Almost Perpendicular Lemma/Reisz Lemma]
    Let $X$ be a normed space and $Y$ be a proper subspace of $X$, then for all $\epsilon>0$, there exists $x\in S_x$ such that $dist(x,Y)\ge 1-\epsilon$
\end{theorem}
\begin{proof}
    Let $\epsilon>0$. Pick a non-zero element $\hat z \in X/Y$ and rescale it such that $1-\epsilon\le\|\hat z\|<1$. Then pick $z \in \hat z$ such that $\|z\|\le 1$. 
    Let $x = \frac{z}{\|z\|}$ such that $x\in S_X$. Then $\hat x=\{a=y+\frac{z}{\|z\|}:y\in Y\}=\frac{\hat z}{\|z\|}$. By definition, $\dist(x,Y)=\|\hat x\|=\|\frac{\hat z}{\|z\|}\|\in[1-\epsilon, 1)$
\end{proof}
\begin{example}
    Consider $X=l_p$ and a sequence $\{e_i\}$. For $i \neq j$, $\|e_i-e_j\|\ge 1$ which means no subsequence of $\{e_i\}$ converges, then $B_X$ is not compact.
\end{example}
\begin{example}
    X is a normed space. $B_X$ is compact iff $X$ is finite dimensional.
    \begin{proof}
        Assume $X$ is finite dimensional. Then $X \simeq l_2^N$ where $N=dim(X)$. $\simeq$ stands for isomorphic, which is surjective isomorphism, in normed space, it means
        the norm of two spaces are equivalent. Then since the topology in metric spaces are induced by metrics, the topology are the same. Then the unit ball in $l_2^N$ is 
        bounded and closed, which means it is compact. Since the topologies induced are the same, the unit ball is also compact in $X$.\\
        Assume $X$ is infinite dimensional and we are to prove $B_X$ is not compact. Pick a point on $S_x$, then $span\{x_1\}$ is a closed proper subspace of $X$. By Reisz's lemma, 
        there exists $x_2$ on $S_X$ such that $\|x_1-x_2\|\ge 1/2$. $span\{x_1, x_2\}$ is a closed proper subspace of $X$, apply the lemma again, there exists $x_3$ on $S_X$
        such that $\|x_1-x_3\|\ge 1/2$, $\|x_2 - x_3\|\ge 1/2$. Proceed this process inductively, then $\{x_n\}$ has no convergent subsequence, then $B_X$ is not compact.
    \end{proof}
\end{example}
\section{Separable Space}
\begin{definition}
A space $X$ is separable if it contains a countable dense subset. A subset $E$ of $X$ is dense if for all open set $O\in X$, $O \cap E\neq \emptyset$.
\end{definition}
$c_0$ is separable. Let $A=\{(q_1,q_2,...,q_m,0,...): q_i \in \Q\}$. $A$ is countable. Left to show $A$ is dense. Pick $x=(x_1,...,x_n,...)\in c_0$ and $\epsilon>0$, we have
$x_n\to 0$. Then there exists $n_0\in \N$ such that $|x_i|<\epsilon/2$ for all $i\ge n_0$. Let $y=(x_1,..,x_{n_0},0,0,...)$. Then $\|x-y\|_\infty < \epsilon/2$. Let $z=\{q_1,...,q_{n_0}\}$ where 
$|x_i-q_i|<\epsilon/2$. Then $\|x-z\|\le\|x-y\|+\|y-z\|<\epsilon$,i.e., $z\in B_\epsilon(x)$ and also $z\in A$ by construction. Therefore $A$ is dense.

$c_0$ is isomorphic to $c$ so $c$ is also separable.

$l_\infty$ is not separable. Let $\alpha$ be a subset of $\N$ and $\mathbb I_\alpha$ be the indicator function. Then $\{\mathbb I_\alpha: \alpha\subset \N\}$ is an 
uncountable set. If there exists an $A\subset l_\infty$ be countable and dense. Then for each $\mathbb I_\alpha$, there exists an $a_\alpha$ such that 
$\|a_\alpha-\mathbb I_\alpha\|<1/3$. Then for $\alpha\neq \beta$, we have
\begin{align*}
    1=\|I_\alpha-I_\beta\|&=\|I_\alpha-a_\alpha+a_\beta - I_\beta+a_\alpha-a_\beta\|\\
    &\le \|I_\alpha-a_\alpha\| + \|a_\beta - I_\beta\|+\|a_\alpha-a_\beta\|\\
    &\le 2/3 + \|a_\alpha-a_\beta\|
\end{align*}
So $\|a_\alpha-a_\beta\|\neq 0 \Rightarrow a_\alpha\neq a_\beta$. Then $\{a_\alpha: \alpha \subset N\}$ is an uncountable subset of $A$.

$L_\infty([0,1])$ is also not separable. We can map every point in $l_\infty$ into $L_\infty([0,1])$. That is define $f_n = \mathbb I_{[\frac{1}{2^{n}}, \frac{1}{2^{n-1}}]}$. 
We then define $T:e_n\to f_n$, i.e., $\forall x \in l_\infty,(Tx)(s)=x_n$ for $s\in [\frac{1}{2^{n}}, \frac{1}{2^{n-1}}]$. Then the uncoutable equidistant set in $l_\infty$ 
is mapped to $L_\infty([0,1])$. Then if a set is dense, then it should be uncountable. $l_1\subset l_2\subset \subset...\subset c\subset l_\infty$ and all of them are 
separable except $l_\infty$.

$c([0,1])$ is separable. Let $A=\{\text{all peice-wise linear functions with rational nodes}\}$. For all $f \in c([0,1])$, $f$ is uniformly continuous. For all $\epsilon>0$, 
there exists $n$ sufficiently large such that all intervals taking the form $[\frac{i}{n}, \frac{i+1}{n}]$ for $i\le n$, the variation of $f$ is less than $\epsilon$.
Then there exists $h\in A$ such that for all $i\in [n]$ and $t\in [\frac{i}{n}, \frac{i+1}{n}]$, $|f(t)-h(t)|<\epsilon$. Then $||f-h||<\epsilon$ and $A$ is dense.
\begin{lemma}
    c([0,1]) is dense in $L_p([0,1])$. 
\end{lemma}
\begin{proof}
    WLOG, we assume $f\ge 0$, otherwise we can replace $f=f^+-f^-$. For all $f\ge 0\in c([0,1])$, let $g_n$ be $f\land n$, then $g_n\to f$.
    Note that $|f-g_n|^p\le 2^p|f|^p$ therefore in $L_p([0,1])$. By Dominated Convergence Theorem, $\lim_{n\to \infty}\int |f-g_n|^p = \int \lim_{n\to \infty}|f-g_n|^p=0$. 
    Hence for all $\epsilon>0$, there exists $M\in \R$ such that $\|f-g\|=\int |f-g|<\epsilon$. By Lusin's theorem, which states that every finite measurable function is an almost 
    continuous function, i.e., there exists $h \in c([0,1])$ such that $g$ agrees with $h$ on a set $A$ with $\mu(A)<\frac{\epsilon}{M}$. WLOG, we assume $h\le M$ otherwise we 
    pay a price of $\epsilon$ as before to truncate it in. Then $\|g-h\|=\int |g-h|\le M\mu(A)=\epsilon$. Finally $\|f-h\|<\|f-g\|+\|g-h\|<2\epsilon$ and $c([0,1])$ is thus dense.
\end{proof}
\begin{theorem}
    For $1\le p <\infty, L_p([0,1])$ is separable.
\end{theorem}
\begin{proof}
    Since $c([0,1])$ is separable in $\|\cdot\|_\infty$ we pick a countable dense set $A$. We claim that $A$ is still a dense set in $L_p([0,1])$ in $\|\cdot\|_p$. 
    Fix $\epsilon>0$. For all $f \in L_p([0,1])$, by previous lemma, there exists  $g \in c([0,1])$ such that $\|f-g\|<\epsilon$. Then we find $h$ in $A$ such that $\|g-h\|_\infty<\epsilon$.
    Then $\|f-g\|<\|f-g\|+\|g-h\|<\|f-g\|+\|g-h\|_\infty<2\epsilon$. The last inequality is due to $\|f\|_p<\|f\|_\infty$ in $L_p([0,1])$.
\end{proof}
\end{document}